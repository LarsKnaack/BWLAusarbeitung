% select subfiles base file
\documentclass[00_ToyotaProduktionssystem.tex]{subfiles}
\begin{document}

\chapter{Entstehung und Geschichte}
\label{chap:ENTSTEHUNG}
\section{Anfänge}
Als Urvater des TPS gilt Sakichi Toyoda. Dieser gründete im Jahr 1918 die Toyoda Spinning and Weaving Company. Er entwickelte den ersten Dampfbetriebenen Webstuhl, welcher in der lage war selbstständig auf einen Fadenriss zu reagieren und sich anzuhalten. So konnte man Produktionsfehlern vorbeugen. Dies legte den Grundstein für die erste Säule des TPS, das Jidoka Prinzip.
\subsection{Das Jidoka Prinzip}
Das Jidoka-Prinzip setzt den Fokus auf Qualität in der  und kann als “Automation mit einer menschlichen Note” beschrieben werden. So gibt es in der gesamten Fertigungslinie Qualitätskontrollen. Tritt nun ein Fehler auf, wird nach dessen Ursache gesucht und notfalls die ganze Produktion angehalten.
\subsubsection{Vorteile}
Der bedeutendste Vorteil des Jikoda Prinzips ist die Vermeidung von Produktionsfehlern. Somit werden Kosten   welche durch Ausschuss, erhöhten Verschleiß der Maschinen oder fehlerhafte Teile in den nachfolgenden Produktionsschritten entstehen eliminiert. Des Weiteren werden Personalkosten verringert, zum Einen durch den Wegfall der manuellen Überwachung der Produktion der Maschinen und den eventuell damit verbundenen Nachbearbeitungsschritt. Außerdem werden Personalkosten auch durch weniger Personaleinsatz bei der Qualitätssicherung verringert.\\
Außerdem bietet das System die Grundlage zur schnellen Fehlerfindung und -eleminierung.
\section{Toyota Motor Corporation}
Sakichis Sohn Kiichiro gründete 1937 die Toyota Motor Corporation und übernahm das Jidoka Konzept. Da in Japan Rohstoffknappheit herrschte und die USA Isolationspolitik gegenüber Japan betrieb führte Kiichiro die zweite Säule des TPS ein, das Just-in-Time Prinzip. Hierdurch konnte man die vorhandenen Ressourcen effizient nutzen. Für die Einführung des Just-in-Time prinzips innerhalb des Unternehmens besuchte Kiichiro Ford Fertigungsstraßen in den USA und orientierte sich an deren Umsetzung.
\subsection{Just-in-Time Prinzip}
Die JIT Produktion, auch bedarfsorientierte Produktion genannt, bezeichnet ein dezentrales Steuerungskonzept bei dem,  wie der Name schon suggeriert, nur die zur Erfüllung der Kundenaufträge notwendigen Mengen an Gütern produziert und es wird vollkommen auf ein Lager verzichtet. Der Verzicht auf Lager betrifft dabei die komplette Produktionslinie, das heißt es gibt auch keine internen Lager und die Produktionssteuerung wird durch ein PULL-System umgesetzt. Anders als beim der PUSH-Steuerung werden beim PULL-Ablauf keine Waren auf Vorrat für den nächsten Produktionsschritt produziert, sondern nur auf Nachfrage des zu beliefernden Produktionsschritts.
\subsection{Das Toyota 3M Modell}
Eine wichtige Rolle bei Just-in-Time spielen auch die so genannten 3M oder Mu´s.
\subsubsection{Mura}
Mura ist die \textbf{Unausgeglichenheit}. Mura können zum Beispiel Unterschiede des Kundenwunsches, Unterschiede in der Prozesszeiten oder Unausgeglichene Arbeitsweise der Mitarbeiter sein. Mura kann zum Beispiel mit Hilfe des Heijunka oder Flow Prinzipes minimiert werden.
\subsubsection{Muda}
Muda ist die \textbf{Verschwendung}. Muda unterteilt sich in Acht Arten: Defekte, Überproduktion, Wartezeiten, ungenutzte Fähigkeiten, Transport, Bestände, Bewegung und Extra Arbeiten. Für jede Art gibt es ein Intrument um diese zu beseitigen: Poke, Yoke, Kanban, Takt Zeit, Single Minute Exchange of Die (SMED) und One-Piece Flow.
\subsubsection{Muri}
Muri ist die \textbf{Überbeanspruchung}. Muri kann entstehen, wenn zu viel Muda(Verschwendung) elminiert ohne das auch Mura (Unausgeglichenheit) minimiert wird. Man unterscheidet zwischen Überbeanspruchung der Maschinen und Überbeanspruchung der Menschen. Eine Überbeanspruchung der Maschinen führt normalerweise zu Störungen und Überbeanspruchung der Menschen zu Krankheitstagen. 
\section{Nachkriegszeit}
Taiichi Ohno, ein junger Toyota Ingenieur der die Produktivität erhöhen sollte reiste 1953 erneut in die USA um die Methoden bei Ford zu studieren. Am meisten beeindruckten ihn die Amerikanischen Supermärkte. Er realisierte, dass es im Grunde Warenlager sind, deren Wareneingänge ziemlich den Waren ausgängen entsprechen. Das liegt daran, da ein Supermarkt keinen Platz für Langzeitlagerungen hat. Mit dieser Erkenntnis entwickelte Ohno nun das KANBAN-Prinzip. In der Nachkriegszeit war das Just-in-Time Prinzip wichtiger denn je, da durch den Krieg viele Ressourcen knapp waren und man die vorhandenen jetzt effizient nutzen musste.
\subsection{KANBAN Konzept}
Angelehnt an den Supermarkt wird jeder Abschnitt eines Prozesses wie ein Lager gesehen. Die nächste Station ist der Kunde und die Aktuelle Station ist der Supermarkt. Das Teil welches an den nächsten Arbeitsschritt weitergereicht wird ist die Ware. Die aktuelle Station muss nun dafür sorgen, dass die entnommene Menge wieder aufgefüllt wird, was dem befüllen der Regale im Supermarkt entspricht.
In der Fertigung z.B findet man das KANBAN Konzept als Zettel, welcher Entnahme- ,Transport- sowie Produktionsinformationen enthält.
Zusätzlich unterscheidet man noch nach Entnahme- und Produktions-Kanban.
Wird der KANBAN richtig umgesetzt, wird keine weitere Produktionsplanung benötigt. Es werden keine nicht benötigten Teile gefertigt sondern nur die, die entnommen werden.
\subsubsection{Vorteile}
Bei der traditionellen Produktionssteuerung wird der komplette Martialbedarf zentral an einer einzigen Stelle vorausgeplant. Dabei gibt es kaum Möglichkeiten, produktionsbedinge Schwankungen des Materialdurchflusses zu kompensieren, was eine ausreichende Lagerung der Produktionsteile nötig macht und damit hohe Kosten verursacht. Dies macht Systeme mit geringer IT-Unterstützung unflexibel und träge und damit weniger konkurrenzfähig.\\
Im Gegensatz dazu bietet Kanban eine schnelle Anpassung an kurzfristige Änderungen des Materialbedarfs, da die Informationsweiterleitung aktuell und direkt vom Verbraucher zum Lieferanten geleitet wird. Außerdem bietet Kanban die Möglichkeit die teilweise sehr komplexen Produktionsabläufe autonom zu verwalten, was den Steuerungsaufwand deutlich reduziert und die Prozesszusammenhänge transparent macht.
\subsubsection{Nachteile}
Da sich die JIT-Produktion stark von herkommlichen Produktionssystemen unterscheidet bedeutet eine Umstellung auf die JIT-Produktion hohen organisatorischen Aufwand. Beispielsweise muss die Produktion auf eine Fließproduktion umgestellt werden, desweiteren müssen die Mitarbeiter verstärkt geschult werden um das neue System ordentlich nutzen zu können. Außerdem ist die PULL-Produktion vor allem für Produktionen mit relativ geringer Variantenvielfalt und relativ konstantem Verbrauch sinnvoll, da nur hier die nötige Standardisierung der Produktionsabläufe gegeben ist. Für Einzelprodukte oder Sonderaufträge ist Kanban demnach ungeeignet. Zuletzt ist eine der Grundvoraussetzungen für eine funktionierende Pull-Produktion eine hohe Qualität der Zwischenprodukte wodurch ein ausgeprägtes Qualitätmanagement-System bestehend aus automatischer Qualitätskontrolle, Selbstkontrolle und Kontrolle durch die Mitarbeiter nötig ist.
\section{Erweiterungen}

\subsection{Kaizen}
\end{document}