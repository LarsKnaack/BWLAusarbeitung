% select subfiles base file
\documentclass[00_ToyotaProduktionssystem.tex]{subfiles}
\begin{document}

\chapter{Entstehung und Geschichte}
\label{chap:ENTSTEHUNG}
\section{Anfänge}
Als Urvater des TPS gilt Sakichi Toyoda. Dieser gründete im Jahr 1918 die Toyoda Spinning and Weaving Company. Er entwickelte den ersten Dampfbetriebenen Webstuhl, welcher in der lage war selbstständig auf einen Fadenriss zu reagieren und sich anzuhalten
\footnote{Vgl. S6,  http://www.pdf.toyota-forklifts-info.de/Broschuere\_TPS.pdf (09.06.2017)}
%\cite[6]{Toyota_Material_Handling}
. So konnte man Produktionsfehlern vorbeugen. Dies legte den Grundstein für die erste Säule des TPS, das Jidoka Prinzip.
\subsection{Das Jidoka Prinzip}
Das Jidoka-Prinzip setzt den Fokus auf Qualität in der  und kann als “Automation mit einer menschlichen Note” beschrieben werden. So gibt es in der gesamten Fertigungslinie Qualitätskontrollen. 
\footnote{Vgl. S10,  http://www.pdf.toyota-forklifts-info.de/Broschuere\_TPS.pdf (09.06.2017)}
%\cite[10]{Toyota_Material_Handling}
 Tritt nun ein Fehler auf, wird nach dessen Ursache gesucht und notfalls die ganze Produktion angehalten.
\subsection{Andon}
Andon ist eine Methode zur Prozessoptimierung und Bestandteil des Toyota Produktionssystems. Ursprünglich handelte es sich bei Andon um eine Methode des visuellen Managements und wurde eingeführt um Funktionen und Abläufe der verschiedenen Produktionslinien zu vermitteln. Inzwischen unterscheidet man zwischen dem Andon-Board, welches Informationen über den Status und eventuell auftretende Probleme der Produktion präsentiert und der Andon-Cord, welche im Falle einer Störung vom zuständigen Mitarbeiter betätigt wird um die Produktionslinie zu stoppen.

\subsubsection{5 W Fragen}
Ist ein Fehler gefunden und die Produktion angehalten, sollen die Mitarbeiter sich nun folgende fünf Fragen stellen:
\begin{itemize}
\item{Was ist genau passiert?}
\item{Wo ist das Problem aufgetreten?}
\item{Wann ereignete sich der Vorfall?}
\item{Wer war daran beteiligt? Wer hat das Problem entdeckt?}
\item{Welche Auswwirkungen sind durch das Problem entstanden?}
\footnote{Vgl.  http://www.lean-production-expert.de/lean-production/5-why.html (09.06.2017)}
%\cite{lean_production_expert}
\end{itemize}
So sind Sie in der Lage, den Fehler schneller zu lokalisieren. Damit wird verhindert, dass nur ein Symptom des Fehlers gefunden wird und nicht die Fehlerursache.


\subsubsection{Vorteile}
Der bedeutendste Vorteil des Jikoda Prinzips ist die Vermeidung von Produktionsfehlern. Somit werden Kosten   welche durch Ausschuss, erhöhten Verschleiß der Maschinen oder fehlerhafte Teile in den nachfolgenden Produktionsschritten entstehen eliminiert. Des Weiteren werden Personalkosten verringert, zum Einen durch den Wegfall der manuellen Überwachung der Produktion der Maschinen und den eventuell damit verbundenen Nachbearbeitungsschritt. Außerdem werden Personalkosten auch durch weniger Personaleinsatz bei der Qualitätssicherung verringert.\\
Außerdem bietet das System die Grundlage zur schnellen Fehlerfindung und -eleminierung.
\section{Toyota Motor Corporation}
Sakichis Sohn Kiichiro gründete 1937 die Toyota Motor Corporation und übernahm das Jidoka Konzept. Da in Japan Rohstoffknappheit herrschte und die USA Isolationspolitik gegenüber Japan betrieb führte Kiichiro die zweite Säule des TPS ein, das Just-in-Time Prinzip
\footnote{Vgl. S6,  http://www.pdf.toyota-forklifts-info.de/Broschuere\_TPS.pdf (09.06.2017)}
%\cite[6]{Toyota_Material_Handling}
. Hierdurch konnte man die vorhandenen Ressourcen effizient nutzen. Für die Einführung des Just-in-Time prinzips innerhalb des Unternehmens besuchte Kiichiro Ford Fertigungsstraßen in den USA und orientierte sich an deren Umsetzung.
\subsection{Just-in-Time Prinzip}
Die JIT Produktion, auch bedarfsorientierte Produktion genannt, bezeichnet ein dezentrales Steuerungskonzept bei dem,  wie der Name schon suggeriert, nur die zur Erfüllung der Kundenaufträge notwendigen Mengen an Gütern produziert und es wird vollkommen auf ein Lager verzichtet. Der Verzicht auf Lager betrifft dabei die komplette Produktionslinie, das heißt es gibt auch keine internen Lager und die Produktionssteuerung wird durch ein PULL-System umgesetzt. Anders als beim der PUSH-Steuerung werden beim PULL-Ablauf keine Waren auf Vorrat für den nächsten Produktionsschritt produziert, sondern nur auf Nachfrage des zu beliefernden Produktionsschritts.

\subsection{Heijunka}
Heijunka heißt übersetzt "Produktionsglättung". Hier versucht man durch genaue abschätzung der künftigen Nachfrage unnötigen Lagerbeständen entgegenzuwirken
\footnote{Vgl.  http://www.panview.nl/de/lean-produktion-lean-basis/das-toyota-3m-modell-muda-mura-muri (09.06.2017)}
%\cite{Management_Circle}
. Heijunka stellt so zu sagen das Gerüst für das Just-in-Time Prinzip bei Toyota dar. Heijunka nutzt z.B Marktforschung, Analyse von Vergangenheits- und Branchendaten sowie Erfassung vergangener Verkaufszahlen um Prognosen für zukünftige Absatzmengen zu erstellen. So kann 
\\
\\
In der Reallität nutzt man so genannte Heijunka-Tafeln. Dort wird aufgelistet wann, wie viel Hergestellt werden muss. Kommt es zu großen abweichungen in der Nachfrage gerät das System ins schleudern. Was gerade am Anfang häufig der Fall ist, da man noch nicht so viele Daten für die Schätzungen hat. 
Ist dies der Fall sollte man den Grund für die Abweichung suchen und möglichst schnell wieder zum Plan zurückkehren. Treten immer weniger Abweichungen auf, so ist man auf dem richtigen weg.

\subsection{Das Toyota 3M Modell}
Eine wichtige Rolle bei Just-in-Time spielen auch die so genannten 3M oder Mu´s.
\subsubsection{Mura}
Mura ist die \textbf{Unausgeglichenheit}. Mura können zum Beispiel Unterschiede des Kundenwunsches, Unterschiede in der Prozesszeiten oder Unausgeglichene Arbeitsweise der Mitarbeiter sein. Mura kann zum Beispiel mit Hilfe des Heijunka oder Flow Prinzipes minimiert werden. \cite{panview}
\subsubsection{Muda}
Muda ist die \textbf{Verschwendung}. Muda unterteilt sich in Acht Arten: Defekte, Überproduktion, Wartezeiten, ungenutzte Fähigkeiten, Transport, Bestände, Bewegung und Extra Arbeiten. Für jede Art gibt es ein Intrument um diese zu beseitigen: Poke, Yoke, Kanban, Takt Zeit, Single Minute Exchange of Die (SMED) und One-Piece Flow. 
\footnote{Vgl.  http://www.panview.nl/de/lean-produktion-lean-basis/das-toyota-3m-modell-muda-mura-muri (09.06.2017)}
%\cite{panview}
\subsubsection{Muri}
Muri ist die \textbf{Überbeanspruchung}. Muri kann entstehen, wenn zu viel Muda(Verschwendung) elminiert ohne das auch Mura (Unausgeglichenheit) minimiert wird. Man unterscheidet zwischen Überbeanspruchung der Maschinen und Überbeanspruchung der Menschen. Eine Überbeanspruchung der Maschinen führt normalerweise zu Störungen und Überbeanspruchung der Menschen zu Krankheitstagen. \\ Diese drei M stehen in Verbindung und sind gewissermaßen voneinander abhängig. Versucht man zum Beispiel nur das Muda zu minimieren ohne das Mura zu minimieren, führt dies zu Muri. Wenn der Kundenwunsch zum Beispiel jeden Tag zwischen 10 und 20 Produkten schwankt, gleichen wir dies durch 8 Bestände aus. So ist der Mitarbeiter in der Lage 15 Produkte am Tag produzieren. Entfernt man diese Bestände, so muss der Mitarbeiter bis zu 20 Produkte produzieren. Dies könnte eine Überlastung des Mitarbeiters oder der Maschine sein. 
\footnote{Vgl.  http://www.panview.nl/de/lean-produktion-lean-basis/das-toyota-3m-modell-muda-mura-muri (09.06.2017)}
%\cite{panview}

\subsection{Taktzeit}
Der Takt gibt an was der Markt benötigt bzw. die Rate der Kundenanfrage. Die Taktzeit Bezeichnet einen Arbeitszyklus, welcher die Kundennachfrage erfüllt. So ist es wichtig, den Arbeitszyklus mit der Nachfrage zu Snychronisieren. Der Produktionsfluss folgt der Taktzeit. Hiermit kann berechnet werden, wie viel Arbeit man schaffen kann. Optimiert man die Taktzeit, läuft man nicht gefahr in Verzug zu geraten oder Überschüsse zu produzieren. So verringert man Verschwendung und Ineffizienz. 
\footnote{Vgl. S9,  http://www.pdf.toyota-forklifts-info.de/Broschuere\_TPS.pdf (09.06.2017)}
%\cite[9]{Toyota_Material_Handling}
\\
Um zu überprüfen obe einzelne Arbeitsschritte mit dem Kundentakt synchron sind verwendet man das so genannte Taktzeitdiagramm. Hiermit kann man die Zykluszeiten einzelner Produktionsschritte miteinander vergleichen und sofort ableßen, welche Schritte den Kundentakt nicht einhalten können. 
\footnote{Vgl. S9,  http://www.lean-production-expert.de/lean-production/taktzeitdiagramm.html (09.06.2017)}
%\cite{lean_production_expert_taktzeit}


\section{Nachkriegszeit}
Taiichi Ohno, ein junger Toyota Ingenieur der die Produktivität erhöhen sollte reiste 1953 erneut in die USA um die Methoden bei Ford zu studieren. Am meisten beeindruckten ihn die Amerikanischen Supermärkte. Er realisierte, dass es im Grunde Warenlager sind, deren Wareneingänge ziemlich den Waren ausgängen entsprechen. Das liegt daran, da ein Supermarkt keinen Platz für Langzeitlagerungen hat. Mit dieser Erkenntnis entwickelte Ohno nun das KANBAN-Prinzip.
\footnote{Vgl. S7,  http://www.pdf.toyota-forklifts-info.de/Broschuere\_TPS.pdf (09.06.2017)}
%\cite[7]{Toyota_Material_Handling}
 In der Nachkriegszeit war das Just-in-Time Prinzip wichtiger denn je, da durch den Krieg viele Ressourcen knapp waren und man die vorhandenen jetzt effizient nutzen musste.
\subsection{KANBAN Konzept}
Angelehnt an den Supermarkt wird jeder Abschnitt eines Prozesses wie ein Lager gesehen. Die nächste Station ist der Kunde und die Aktuelle Station ist der Supermarkt. Das Teil welches an den nächsten Arbeitsschritt weitergereicht wird ist die Ware. Die aktuelle Station muss nun dafür sorgen, dass die entnommene Menge wieder aufgefüllt wird, was dem befüllen der Regale im Supermarkt entspricht.
In der Fertigung z.B findet man das KANBAN Konzept als Zettel, welcher Entnahme- ,Transport- sowie Produktionsinformationen enthält.
Zusätzlich unterscheidet man noch nach Entnahme- und Produktions-Kanban.
Wird der KANBAN richtig umgesetzt, wird keine weitere Produktionsplanung benötigt. Es werden keine nicht benötigten Teile gefertigt sondern nur die, die entnommen werden. 
\footnote{Vgl. S7,  http://www.pdf.toyota-forklifts-info.de/Broschuere\_TPS.pdf (09.06.2017)}
%\cite[7]{Toyota_Material_Handling}

\subsection{Beispiel}
Beim folgenden Beispiel wird das JIT-Prinzip mittels Kanban erläutert, da Kanban die bekannteste Umsetzung eines PULL-Prinzips ist.
In einem Automobilzulieferer werden Mittelkonsolen hergestellt. Für die Endmontage wird die Chromverkleidung mit der spritzgegossenen Konsole verklebt. Alle Produkte und Materialien werden in Kanbanbehältern mit Kanbankarten der Losgröße 5 aufbewahrt und transportiert von denen in jeder Abteilung jeweils 2 Behälter stehen. Der Klebstoff wird allerdings vom Einkauf direkt bestellt und fällt damit nicht unter den Kanbanablauf und wird deshalb in folgendem Beispiel ignoriert.

\paragraph{Schritt 1}
Die Endmontage erhält nun den Kundenauftrag 7 Mittelkonsolen herzustellen und entnimmt den Vorproduktionen jeweils einen Kanbanbehälter. Und stellt nach dem Verbrauch der Teile die leeren Behälter wieder in die Vorproduktionen zurück.

\paragraph{Schritt 2}
Durch die zurückgestellten Behälter und die daran angebrachten Kanbankarten wird den Vorproduktionen signalisiert, diese Behälter wieder zu füllen. Für die Produktion der Teile wird eventuell auch ein Vorprodukt benötigt was mit dem selben Prinzip hergestellt wird wie in Schritt 1 beschrieben.

\paragraph{Schritt 3}
Die Endmontage nimmt sich jeweils einen weiteren Behälter aus den Vorproduktionen. Da diese Behälter aber nicht vollständig geleert werden, bleiben sie in der Montage stehen.\\
Währenddessen haben die Vorproduktionen ihre Behälter wieder gefüllt und bereitgestellt.

\paragraph{Schritt 4}
Die Endmontage erhält nun den Kundenauftrag 6 weitere Teile herzustellen und fängt mit der Produktion an. Nach 3 gefertigten Teilen werden die leeren Behälter wieder in die Vorproduktionen zurückgestellt und neue Behälter entnommen.

\paragraph{Weitere Schritte}
Mit dem nächsten Kundenauftrag, sofern dieser eine Losgröße > 2 hat wird der Vorrat der Behälter in der Endmontage aufgebraucht sein und der Ablauf beginnt wieder von vorne.\\
Idealerweise wird der benötigte Klebstoff auch mit zwei Behältern der Endmontage zur Verfügung gestellt. Durch den Einsatz eines Umlagerungskanbans könnte der jeweilige leere Behälter vom innerbetrieblichen Transport in einem zentralen Lager wieder aufgefüllt werden.

\subsubsection{Vorteile}
Bei der traditionellen Produktionssteuerung wird der komplette Martialbedarf zentral an einer einzigen Stelle vorausgeplant. Dabei gibt es kaum Möglichkeiten, produktionsbedinge Schwankungen des Materialdurchflusses zu kompensieren, was eine ausreichende Lagerung der Produktionsteile nötig macht und damit hohe Kosten verursacht. Dies macht Systeme mit geringer IT-Unterstützung unflexibel und träge und damit weniger konkurrenzfähig.\\
Im Gegensatz dazu bietet Kanban eine schnelle Anpassung an kurzfristige Änderungen des Materialbedarfs, da die Informationsweiterleitung aktuell und direkt vom Verbraucher zum Lieferanten geleitet wird. Außerdem bietet Kanban die Möglichkeit die teilweise sehr komplexen Produktionsabläufe autonom zu verwalten, was den Steuerungsaufwand deutlich reduziert und die Prozesszusammenhänge transparent macht.
\subsubsection{Nachteile}
Da sich die JIT-Produktion stark von herkommlichen Produktionssystemen unterscheidet bedeutet eine Umstellung auf die JIT-Produktion hohen organisatorischen Aufwand. Beispielsweise muss die Produktion auf eine Fließproduktion umgestellt werden, desweiteren müssen die Mitarbeiter verstärkt geschult werden um das neue System ordentlich nutzen zu können. Außerdem ist die PULL-Produktion vor allem für Produktionen mit relativ geringer Variantenvielfalt und relativ konstantem Verbrauch sinnvoll, da nur hier die nötige Standardisierung der Produktionsabläufe gegeben ist. Für Einzelprodukte oder Sonderaufträge ist Kanban demnach ungeeignet. Zuletzt ist eine der Grundvoraussetzungen für eine funktionierende Pull-Produktion eine hohe Qualität der Zwischenprodukte wodurch ein ausgeprägtes Qualitätmanagement-System bestehend aus automatischer Qualitätskontrolle, Selbstkontrolle und Kontrolle durch die Mitarbeiter nötig ist.
\section{Erweiterungen}

\subsection{Kaizen}
Kaizen stellt die grundlegende Philosophie des Unternehmens dar, ständig nach Verbesserung zu streben. Für Kaizen muss allerdings Klarheit darüber herrschen, was erreicht werden soll. 


\subsubsection{Verbesserung durch 5S}
Die 5 S dienen zur Beseitigung von Verschwendung. Also von unnötiger Arbeitszeit.

\begin{itemize}
\item{SEIRI - Sortieren} Hier soll man schauen, ob alles Sortiert ist. So dass man alles auf Anhieb findet und in greifbarer Nähe ist. 
\item{SEITON - Ordnungsliebe} Man sollte die Anordnung der Gegenstände an seinem Arbeitsplatz überdenken. Beschriftungen der Orte wo Gegenstände abgelegt werden, sollten beschriftet werden.
\item{SEISO - Sauberkeit} Hierbei ist man aufgefordert den Arbeitsplatz aufzuräumen. Unnütze Dinge sollten entfernt werden. 
\item{SEIKETSU - Ordnungsregeln/ Standardisierung} Alles was man erarbeitet hat, sollte man aufschreiben und auf ähnliche Arbeitsplätze/ Arbeitsschritte anwenden.  Man sollte hierzu einheitliche Markierungen und Kennzeichnungen verwenden. 
\item{SHITSUKE - Disziplin} Eine Vorraussetzung dafür, dass das erarbeitete dauerhaft umgesetzt wird ist die Selbstdisziplin. Ohne Selbstdisziplin verfällt man schnell in alte Muster und setzt nicht das neu erarbeitete um.
\footnote{Vgl.  https://www.gruenderszene.de/allgemein/prozessoptimierung-mit-der-5s-methode(09.06.2017)}
%\cite{Gruenderszene}
\end{itemize}

Die 5 S können gut als Einstieg in die Prozessoptimierung genommen werden, denn es lassen sich mit wenig Aufwand schnell Erfolge erzielen. Sie dienen auch als Denkanstoß für die Mitarbeiter.

\end{document}