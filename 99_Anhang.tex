% select subfiles base file
\documentclass[00_Praxissemesterbericht.tex]{subfiles}
\begin{document}

%\phantomsection
\chapter*{Anhang}
\label{chap:APPENDIX}
\addcontentsline{toc}{chapter}{Anhang}
%\setcounter{chapter}{0}
\addtocounter{chapter}{1}
\setcounter{section}{0}

\section{Abbildungen}
\label{chap:APPENDIX_FIGURES}

\begin{figure}[!htbp]
\caption{JGiven HTML Report}
\label{fig:JGIVEN_HTML}
\includegraphics[width=\textwidth]{media/pictures/JGiven_HTML_Report.png}
\begin{center}
Quelle \cite{JGiven}
\end{center}
\end{figure}

\begin{figure}[!htbp]
\caption{JGiven Konsolen Report}
\label{fig:JGIVEN_TEXT_REPORT}
\begin{verbatim}
Scenario: a pancake can be fried out of an egg milk and flour

  Given an egg
    And some milk
    And the ingredient flour
   When the cook mangles everything to a dough
    And the cook fries the dough in a pan
   Then the resulting meal is a pan cake
\end{verbatim}
\end{figure}

\begin{figure}[!htbp]
\caption{Git Workflow}
\label{fig:GIT_WORKFLOW}
\includegraphics[width=\textwidth]{media/pictures/Git_Workflow.png}
Erläuterung:
\begin{itemize}
\item[A]{Beginn eines Features: Ein neuer Branch wird aus dem "{}stable"{} Branch erstellt}
\item[B]{Vorbereitung des Reviews: Letzte Änderungen des "{}stable"{} Branches in den aktuellen Feature Branch mergen}
\item[C]{Review: Code überprüfen, zusätzliche Tests ausführen}
\item[D]{Tests schlagen fehl: Entwickler muss den Fehler beheben, andere Entwickler werden nicht beeinträchtigt}
\item[E]{Review abgenommen: Änderungen in "{}stable"{} Branch mergen (mit Jenkins Job)}
\item[F]{Release: Release wird mit Gradle gebaut}
\end{itemize}
\end{figure}

\begin{figure}[!htbp]
\caption{Git Workflow Alternative}
\label{fig:GIT_WORKFLOW_ALTERNATIVE}
\includegraphics[width=\textwidth]{media/pictures/Git_Workflow_new.png}
\end{figure}
\clearpage

\section{Code Beispiele}
\label{sec:CODE}

\begin{lstlisting}[language=xml, frame=single, caption=Spring Konfiguration XML (Quelle{\cite{Spring_CONFIG}}), captionpos=t, breaklines=true, numbers=left, showstringspaces=false, tabsize=2, label=lst:SPRING_CONFIG_XML]
<?xml version="1.0" encoding="UTF-8"?>

<beans xmlns="http://www.springframework.org/schema/beans"
    xmlns:xsi="http://www.w3.org/2001/XMLSchema-instance"
    xmlns:context="http://www.springframework.org/schema/context"
    xsi:schemaLocation="http://www.springframework.org/schema/beans
    http://www.springframework.org/schema/beans/spring-beans-3.0.xsd
    http://www.springframework.org/schema/context
    http://www.springframework.org/schema/context/spring-context-3.0.xsd">

   <context:annotation-config/>

   <!-- Definition for spellChecker bean -->
   <bean id="spellChecker" class="com.tutorialspoint.SpellChecker">
   </bean>

</beans>
\end{lstlisting}

\begin{lstlisting}[language=java, frame=single, caption=Spring Konfiguration Annotation (Quelle{\cite{Spring_CONFIG}}), captionpos=t, breaklines=true, numbers=left, showstringspaces=false, tabsize=2, label=lst:SPRING_CONFIG_ANNOTATION]
public class TextEditor {
   @Autowired
   private SpellChecker spellChecker;
}
\end{lstlisting}
\clearpage
\begin{lstlisting}[language=java, frame=single, caption = Spring Webservice Beispiel (Quelle{\cite[vgl.]{Spring_REST}}), captionpos=t, breaklines=true, showstringspaces=false, numbers = left, tabsize = 2, label=lst:SPRING_WS_JAVA]
@RequestMapping("/greeting")
public Greeting greeting(@RequestParam(value="name", defaultValue="World") String name){
	return "Hello " + name;
}
\end{lstlisting}

\begin{lstlisting}[language=java, frame=single, caption = JGiven Java Beispiel (Quelle{\cite{JGiven}}), captionpos=t, breaklines=true, showstringspaces=false, numbers = left, tabsize = 2, label=lst:JGIVEN_JAVA]
@Test
public void a_pancake_can_be_fried_out_of_an_egg_milk_and_flour() {
    given().an_egg().
        and().some_milk().
        and().the_ingredient( "flour" );

    when().the_cook_mangles_everthing_to_a_dough().
        and().the_cook_fries_the_dough_in_a_pan();

    then().the_resulting_meal_is_a_pan_cake();
}
\end{lstlisting}

\begin{lstlisting}[language=java, frame=single, caption = JGiven Steps Beispiel (Quelle{\cite{JGiven_Github}}), captionpos=t, breaklines=true, showstringspaces=false, numbers = left, tabsize = 2, label=lst:JGIVEN_JAVA_STEPS]
public GivenIngredients an_egg() {
	return the_ingredient( "egg" );
}
public GivenIngredients the_ingredient( String ingredient ) {
	ingredients.add( ingredient );
	return this;
}
public GivenIngredients some_milk() {
	return the_ingredient( "milk" );
}
\end{lstlisting}

\clearpage

\end{document}