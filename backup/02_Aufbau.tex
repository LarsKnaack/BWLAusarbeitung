% select subfiles base file
\documentclass[00_ToyotaProduktionssystem.tex]{subfiles}
\begin{document}

\chapter{Aufbau des Produktionssystems}
\label{chap:AUFBAU}
Das Toyota Produktionssystem besteht aus zwei Konzepten, dem Jidoka Prinzip und dem Just-In-Time-Prinzip. Diese beiden Konzepte stehen auf dem Fundament jegliche Verschwendung zu beseitigen und haben als Ziel eine hohe Produktivität bei gleichzeitiger höchster Produktqualität und pünktlicher Lieferung.

\section{Jikoda Prinzip}
Das Jidoka\footnote{jap. autonome Automation} Prinzip beruht auf der Idee des selbst stoppenden Webstuhls von Toyoda Sakichi. Ziel dieses Prinzips ist eine hundertprozentige Kontrolle der gefertigten Produkte. Anders als bei herkömmlichen Produktionssystemen wird beim Jikoda Prinzip diese Kontrolle nicht im Nachhinein sondern während des Prozesses durchgeführt. Dafür werden Maschinen mit den benötigten Sensoren ausgestattet.

\subsection{Andon}
Andon ist eine Methode zur Prozessoptimierung und Bestandteil des Toyota Produktionssystems. Ursprünglich handelte es sich bei Andon um eine Methode des visuellen Managements und wurde eingeführt um Funktionen und Abläufe der verschiedenen Produktionslinien zu vermitteln. Inzwischen unterscheidet man zwischen dem Andon-Board, welches Informationen über den Status und eventuell auftretende Probleme der Produktion präsentiert und der Andon-Cord, welche im Falle einer Störung vom zuständigen Mitarbeiter betätigt wird um die Produktionslinie zu stoppen.

\subsection{Vorteile}
Der bedeutendste Vorteil des Jikoda Prinzips ist die Vermeidung von Produktionsfehlern. Somit werden Kosten   welche durch Ausschuss, erhöhten Verschleiß der Maschinen oder fehlerhafte Teile in den nachfolgenden Produktionsschritten entstehen eliminiert. Des Weiteren werden Personalkosten verringert, zum Einen durch den Wegfall der manuellen Überwachung der Produktion der Maschinen und den eventuell damit verbundenen Nachbearbeitungsschritt. Außerdem werden Personalkosten auch durch weniger Personaleinsatz bei der Qualitätssicherung verringert.\\
Außerdem bietet das System die Grundlage zur schnellen Fehlerfindung und -eleminierung.
\subsection{Nachteile}

\section{Just-In-Time Prinzip}
Die zweite Säule des Toyota Produktionssystems ist neben dem Jikoda Prinzip die Just-In-Time Produktion. Die JIT Produktion, auch bedarfsorientierte Produktion genannt, bezeichnet ein dezentrales Steuerungskonzept bei dem,  wie der Name schon suggeriert, nur die zur Erfüllung der Kundenaufträge notwendigen Mengen an Gütern produziert und es wird vollkommen auf ein Lager verzichtet. Der Verzicht auf Lager betrifft dabei die komplette Produktionslinie, das heißt es gibt auch keine internen Lager und die Produktionssteuerung wird durch ein PULL-System umgesetzt. Anders als beim der PUSH-Steuerung werden beim PULL-Ablauf keine Waren auf Vorrat für den nächsten Produktionsschritt produziert, sondern nur auf Nachfrage des zu beliefernden Produktionsschritts.

\subsection{Kanban}
Eine Umsetzung des Pull-Prinzips ist der Einsatz von Kanban. Kanban ist eine Methode zur dezentralen Bedarfs- und Produktionssteuerung bei dem der Verbraucher das für die Produktion notwendige Material wie im Supermarkt beim Lieferanten entnimmt. Das ursprüngliche Kanban-System wurde 1947 in der Toyota Motor Corporation entwickelt. Die Materialien werden in sog. Losgrößen oder Kanban-Mengen bemessen, hat ein Verbraucher eine Losgröße verbraucht, signalisiert er dem Lieferanten durch eine Kanban\footnote{jap. Karte}-Karte, dass eine neue Menge produziert werden soll. Die Größe der Kanban-Menge ist immer gleich, kann jedoch von Betrieb zu Betrieb variieren. Im Idealfall beträgt die Kanban-Menge genau 1, womit ein sog. One-Piece-Flow erreicht wird, was die Kosten durch Lagerung minimiert und eine maximale Wertschöfung ermöglicht.

\subsection{Vorteile}
Bei der traditionellen Produktionssteuerung wird der komplette Martialbedarf zentral an einer einzigen Stelle vorausgeplant. Dabei gibt es kaum Möglichkeiten, produktionsbedinge Schwankungen des Materialdurchflusses zu kompensieren, was eine ausreichende Lagerung der Produktionsteile nötig macht und damit hohe Kosten verursacht. Dies macht Systeme mit geringer IT-Unterstützung unflexibel und träge und damit weniger konkurrenzfähig.\\
Im Gegensatz dazu bietet Kanban eine schnelle Anpassung an kurzfristige Änderungen des Materialbedarfs, da die Informationsweiterleitung aktuell und direkt vom Verbraucher zum Lieferanten geleitet wird. Außerdem bietet Kanban die Möglichkeit die teilweise sehr komplexen Produktionsabläufe autonom zu verwalten, was den Steuerungsaufwand deutlich reduziert und die Prozesszusammenhänge transparent macht.

\subsection{Nachteile}
Da sich die JIT-Produktion stark von herkommlichen Produktionssystemen unterscheidet bedeutet eine Umstellung auf die JIT-Produktion hohen organisatorischen Aufwand. Beispielsweise muss die Produktion auf eine Fließproduktion umgestellt werden, desweiteren müssen die Mitarbeiter verstärkt geschult werden um das neue System ordentlich nutzen zu können. Außerdem ist die PULL-Produktion vor allem für Produktionen mit relativ geringer Variantenvielfalt und relativ konstantem Verbrauch sinnvoll, da nur hier die nötige Standardisierung der Produktionsabläufe gegeben ist. Für Einzelprodukte oder Sonderaufträge ist Kanban demnach ungeeignet. Zuletzt ist eine der Grundvoraussetzungen für eine funktionierende Pull-Produktion eine hohe Qualität der Zwischenprodukte wodurch ein ausgeprägtes Qualitätmanagement-System bestehend aus automatischer Qualitätskontrolle, Selbstkontrolle und Kontrolle durch die Mitarbeiter nötig ist.

\end{document}