% select subfiles base file
\documentclass[00_ToyotaProduktionssystem.tex]{subfiles}
\begin{document}
Wie man anhand des Jidoka Prinzipes schon erkennen kann, steht beim Toyota Produktionsprinzip der Mitarbeiter und nicht die Maschine im Mittelpunkt. Der Mitarbeiter wird mit Respekt behandelt und hat das Recht auf eine Sinnvolle Aufgabe. Dies erfordert jedoch ein hohes Maß an Disziplin und Effizienz. Schaut man sich einmal das Henry Ford Prinzip an, so wird man Feststellen, dass hier dem Mitarbeiter jegliche Verantwortung abgenommen wird. Hier werden die Arbeitsabläufe so minimalisiert, dass die Mitarbeiter fast keine Fehler mehr machen können. Der einzige Grundsatz lautet "Beweg das Blech" und der Mitarbeiter muss nicht mehr selbst denken. So führt Routine im Arbeitsalltag nachweislich zu einem Rückgang der geistigen Leistungsfähigkeit \footnote{http://www.fr.de/wissen/gehirntraining-job-routine-schlecht-fuer-geistige-leistungsfaehigkeit-a-356825, (09.06.2017)}. So hat der Mitarbeiter hier auch nicht das Gefühl, etwas bewirken zu können sondern eher als Werkzeug. So stärkt das Toyota Produktion das Betriebsklima sowie die Unternehmenskultur.
\end{document}